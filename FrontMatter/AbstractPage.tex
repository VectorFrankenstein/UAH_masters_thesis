% The top of your abstract will fill out automatically once you fill in the required fields on the main.tex file. In this file, you will provide your abstract body. Type your abstract body at the bottom of this page directly below the \doublespacing command.

\chapter{\texorpdfstring{\MakeUppercase{Abstract}}{Abstract}}
     \begin{center}
        \large
        \singlespacing
        \textbf{\thesistitle}\\
        \vspace{0.5cm}
        \large
        \textbf{\studentname}\\
        \vspace{0.5cm}
        \normalsize
        \ifdefined\thesis
        \textbf{A thesis submitted in partial fulfillment of the requirements \\for the degree of \degree}\\  
        \else
        \ifdefined\dissertation
        \textbf{A dissertation submitted in partial fulfillment of the requirements \\for the degree of \degree}\\ 
        \else
        \textbf{Please identify this document as either a thesis or dissertation on the main.tex in the section at the top that must be filled out.}\\
    \fi
    \fi
        \vspace{1cm}
        \textbf{\department}
        
        \vspace{0.25cm}

        \ifdefined\jointuni
        \textbf{The University of Alabama in Huntsville and  \jointuni}
        \else
        \textbf{The University of Alabama in Huntsville}
    \fi

        
        \vspace{0.1cm}
        \textbf{\gradmonth\ \gradyear}
        


    \end{center}
\vspace{0.1cm}

%****************************************************
%Enter the body of your abstract below. Remember there is a 150 word limit!
%****************************************************
\doublespacing
Green plants are ubiquitous and an essential part of the biosphere. There are two different types of reproductive life-cycles in green plants: heterosporous and homosporous. Within the ubiquity of green plants, heterosporous plants are the ones people generally see more often in day-to-day life and tend to be the source of staple foods. Heterosporous reproduction has evolved at least 11 times from ancestral homosporous lineages. There appear to be selective advantages to heterospory. However, more perplexing is the correlation between reproductive life cycle and chromosome numbers across green plants. Here I have used some modern data, tools and methodology to explore the evolution of heterospory through the avenue of the mysterious correlation between plant reproductive life-cycle and chromosome numbers. I explored why heterosporous plants have fewer (on average about one quarter) chromosomes compared to homosporous species.  I found that recent developments in computational phylogenetic methodologies have opened an avenue towards exploring the evolution of traits in organisms. I found 19 gene families with the potential to explain the correlation between reproductive life cycle in green plants and chromosome numbers. This is a significant improvement compared to the fact that before I started this project, there was no known data on the role of gene families with respect to the evolution of reproductive life cycles in plants and the morphological and cytological changes associated with it. The discovery of these gene families provides a potential avenue for gain-of-function functional genomics research. Such studies could provide further insights that could help explain the evolution of traits in green plants, which are a crucial part of the surrounding ecosystems.
\clearpage

