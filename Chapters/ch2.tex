\chapter{Chapter 2. Research Objectives}

Despite the crucial role played by heterospory in the evolution of terrestrial plant life, the mechanisms that underlie transitions to heterospory have remained unclear \cite{Kinosian2022-uf}. In addition, the correlation between heterospory and the drop in chromosome numbers remains unexplored. Why are changes in modes of reproduction and chromosome numbers associated? Are there parallels between the genetic factors such as expansion and contractions of Copy Number Variation (CNVs) in gene families, or selection on specific genes that underlie the transitions? Are the CNVs and selection in the gene families similar, opposite, or completely unrelated among plant species that arise from the nodes leading to heterospory?
Convergent evolution in gene families within independently derived heterosporous lineages could provide evidence for the mechanisms by which heterospory evolved in land plants. Such changes may include expansions or contractions of gene families or changes in the rates of nucleotide substitution that reflect selection on specific gene family members. I hypothesize that the independent origins of heterospory involve selection and/or copy number variation on the same gene families/pathways that could suggest neofunctionalization i.e. when a gene copy takes on a completely new metabolic or biological role or subfunctionalization i.e. diversification of roles of gene copies. . Models of nucleotide substitution rates may detect trends in the selection of specific gene families that underlie transitions to heterospory, or other genomic changes undergoing similar selection rates could be behind the recurrent evolution of heterosporous plants.
Associating changes in gene copy number or selection with these transitions will not explain the causation behind the transitions to heterospory; that is a task for gain-of-phenotype research. However, it will improve our capacity to circumscribe more specific hypotheses to test for potential causes behind the correlation between the transition to heterospory and a reduction in chromosome number. This strictly exploratory study aims to find potential gene families that affect both reproduction and meiosis (and by extension chromosome numbers), two traits that are directly linked by the reproductive life cycle wherein spores are produced by meiosis. Although this study is exploratory in nature, as more homosporous species become genetically transformable, the discovery of potential candidate gene families showing signs of selection on heterosporous nodes could provide novel insights through gene editing and functional validation. 